\documentclass[11pt,paper=a4,answers]{exam}
\usepackage{graphicx,lastpage}
\usepackage{upgreek}
\usepackage{censor}
\usepackage{enumitem}

\censorruledepth=-.2ex
\censorruleheight=.1ex
\hyphenpenalty 10000
\usepackage[paperheight=10.5in,paperwidth=8.27in,bindingoffset=0in,left=0.8in,right=1in,
top=0.7in,bottom=1in,headsep=.5\baselineskip]{geometry}
\flushbottom
\usepackage[normalem]{ulem}
\renewcommand\ULthickness{2pt}   %%---> For changing thickness of underline
\setlength\ULdepth{1.5ex}%\maxdimen ---> For changing depth of underline
\renewcommand{\baselinestretch}{1}
\pagestyle{empty}

\pagestyle{headandfoot}
\headrule
\newcommand{\continuedmessage}{%
\ifcontinuation{\footnotesize Question \ContinuedQuestion\ continues\ldots}{}%
 }
 
\runningheader{\footnotesize BISC}
{\footnotesize BISC 104 --- Lab}
{\footnotesize Page \thepage\ of \numpages}
\footrule
\footer{\vspace*{5pt} 
}
{}
{\ifincomplete{\footnotesize Question \IncompleteQuestion\ continues
on the next page\ldots}{\iflastpage{\footnotesize End of exam}{\footnotesize Please go        on to the next page\ldots}}}

\usepackage{cleveref}
\crefname{figure}{figure}{figures}
\crefname{question}{question}{questions}
%==============================================================
\begin{document}

%% \thispagestyle{empty}

\noindent
\begin{minipage}[l]{.5\textwidth}%
\noindent
Name: \underline{\hspace{7cm}}
%\includegraphics[width=1.5\textwidth]{123}
\end{minipage}
\hfill
\begin{minipage}[r]{0.22\textwidth}%
\begin{center}
{%\Large Name of University \\[2pt]
\large BISC 104 -- Quiz 5\\[2pt]} %{(\small Code: Math-506)}  \par}
%  \vspace{0.5cm}
\end{center}
\end{minipage}
\par
\noindent
\uline{Time: 5 minutes   \hfill \normalsize\emph{\underline{Session 07}} \hfill        Maximum Points: 07}
\begin{questions}

\pointsinrightmargin
\pointsdroppedatright
\marksnotpoints
%\marginpointname{mark}
\pointpoints{mark}{marks}
\pointformat{\boldmath\themarginpoints}
\bracketedpoints
\question[01]
The ability of heart to trigger its own contractions is called \underline{\hspace{3cm}}.
\label{Q:perunit}
\droppoints

\question[01]
\label{Q:perunit}
Symphathetic nervous system \underline{\hspace{3cm}} heart rate.
\droppoints
\begin{enumerate}[label=\alph*]
\item increases
\item decreases
\end{enumerate}


\question[01]
\label{Q:perunit}
Parasymphathetic nervous system \underline{\hspace{3cm}} heart rate.
\droppoints
\begin{enumerate}[label=\alph*]
\item increases
\item decreases
\end{enumerate}


\question[01]
\label{Q:perunit}
Vagus nerves are part of the \underline{\hspace{3cm}} nervous system.
\droppoints
\begin{enumerate}[label=\alph*]
\item Parasymphathetic
\item Symphathetic
\end{enumerate}


\question[01]
\label{Q:perunit}
Sinoatrial(SA) node generates action potentials at a frequency of 100 beats per minute even though the observed average heart beat normally is around 70 beats per minute. This is result of the push-pull system we discussed. In terms of degree of influence, which of the following is \textbf{true}:
\droppoints
\begin{enumerate}[label=\alph*]
\item Parasymphathetic nervous system has slightly more influence over symphathetic
\item Symphathetic nervous system has slightly more influence over parasymphathetic
\item Both systems are equally influential
\item None of the above
\end{enumerate}

\question[02]
\label{Q:perunit}
In your activity 1, you characterized the refractory period of the heart. This involved applying stimuli at different points of time. The reponse changed depending on 'where'/'when' you applied the stimulus. The refractory period lasted between:
\droppoints

\begin{enumerate}[label=\alph*]
\item Beginning of articular contraction to beginning of ventricular contraction
\item Beginning of ventricular contraction to peak of ventricular contraction
\item Peak of ventricular contraction to end of ventricular relaxation
\item None of the above
\end{enumerate}

\end{questions}
\begin{center}
\rule{\textwidth}{1pt}
\end{center}
\end{document} 