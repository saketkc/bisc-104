\documentclass[11pt,paper=a4,answers]{exam}
\usepackage{graphicx,lastpage}
\usepackage{upgreek}
\usepackage{censor}
\usepackage{enumitem}

\censorruledepth=-.2ex
\censorruleheight=.1ex
\hyphenpenalty 10000
\usepackage[paperheight=10.5in,paperwidth=8.27in,bindingoffset=0in,left=0.8in,right=1in,
top=0.7in,bottom=1in,headsep=.5\baselineskip]{geometry}
\flushbottom
\usepackage[normalem]{ulem}
\renewcommand\ULthickness{2pt}   %%---> For changing thickness of underline
\setlength\ULdepth{1.5ex}%\maxdimen ---> For changing depth of underline
\renewcommand{\baselinestretch}{1}
\pagestyle{empty}

\pagestyle{headandfoot}
\headrule
\newcommand{\continuedmessage}{%
\ifcontinuation{\footnotesize Question \ContinuedQuestion\ continues\ldots}{}%
 }
 
\runningheader{\footnotesize BISC}
{\footnotesize BISC 104 --- Lab}
{\footnotesize Page \thepage\ of \numpages}
\footrule
\footer{\vspace*{5pt} 
}
{}
{\ifincomplete{\footnotesize Question \IncompleteQuestion\ continues
on the next page\ldots}{\iflastpage{\footnotesize End of exam}{\footnotesize Please go        on to the next page\ldots}}}

\usepackage{cleveref}
\crefname{figure}{figure}{figures}
\crefname{question}{question}{questions}
%==============================================================
\begin{document}

%% \thispagestyle{empty}

\noindent
\begin{minipage}[l]{.5\textwidth}%
\noindent
Name: \underline{\hspace{7cm}}
%\includegraphics[width=1.5\textwidth]{123}
\end{minipage}
\hfill
\begin{minipage}[r]{0.22\textwidth}%
\begin{center}
{%\Large Name of University \\[2pt]
\large BISC 104 -- Quiz 3\\[2pt]} %{(\small Code: Math-506)}  \par}
%  \vspace{0.5cm}
\end{center}
\end{minipage}
\par
\noindent
\uline{Time: 5 minutes   \hfill \normalsize\emph{\underline{Session 04}} \hfill        Maximum Points: 07}
\begin{questions}

\pointsinrightmargin
\pointsdroppedatright
\marksnotpoints
%\marginpointname{mark}
\pointpoints{mark}{marks}
\pointformat{\boldmath\themarginpoints}
\bracketedpoints
\question[01]
\label{Q:perunit}
A motor unit consists of \underline{\hspace{3cm}} neuron.
\droppoints

\question[01]
\label{Q:perunit}
Strength of striated muscle is \underline{\hspace{3cm}} proportional to number of motor units activated.
\droppoints
\begin{enumerate}[label=\alph*]
\item directly
\item inversely
\end{enumerate}

\question[01]
Area under curve for the potential generated \underline{\hspace{3cm}} as the force applied by muscle  
increases.
\droppoints
\begin{enumerate}[label=\alph*]
\item increases
\item decreases
\item remains constant
\end{enumerate}


\question[01]
Relation between absolute integral under EMG signals and the absolute integral under the muscle contraction
is \underline{\hspace{3cm}}
\droppoints
\begin{enumerate}[label=\alph*]
\item non-linear
\item linear
\item constant
\end{enumerate}

\question[02]
The amplitude of the EMG signal and the force of contraction, as measured by the absolute
integrals, increase because:
\droppoints
\begin{enumerate}[label=\alph*]
\item finite number of fibers are firing more often
\item more number of fibers recruited to fire
\item both
\end{enumerate}

\question[01]
Refractory period is defined as the short period immediately following one stimulus when the muscles do not respond to the second stimulus. Muscle fibers do not have refractory period. (True/False)
\droppoints

\end{questions}
\begin{center}
\rule{\textwidth}{1pt}
\end{center}
\end{document} 