\documentclass[11pt,paper=a4,answers]{exam}
\usepackage{graphicx,lastpage}
\usepackage{upgreek}
\usepackage{censor}
\usepackage{enumitem}

\censorruledepth=-.2ex
\censorruleheight=.1ex
\hyphenpenalty 10000
\usepackage[paperheight=10.5in,paperwidth=8.27in,bindingoffset=0in,left=0.8in,right=1in,
top=0.7in,bottom=1in,headsep=.5\baselineskip]{geometry}
\flushbottom
\usepackage[normalem]{ulem}
\renewcommand\ULthickness{2pt}   %%---> For changing thickness of underline
\setlength\ULdepth{1.5ex}%\maxdimen ---> For changing depth of underline
\renewcommand{\baselinestretch}{1}
\pagestyle{empty}

\pagestyle{headandfoot}
\headrule
\newcommand{\continuedmessage}{%
\ifcontinuation{\footnotesize Question \ContinuedQuestion\ continues\ldots}{}%
 }
 
\runningheader{\footnotesize BISC}
{\footnotesize BISC 104 --- Lab}
{\footnotesize Page \thepage\ of \numpages}
\footrule
\footer{\vspace*{5pt} 
}
{}
{\ifincomplete{\footnotesize Question \IncompleteQuestion\ continues
on the next page\ldots}{\iflastpage{\footnotesize End of exam}{\footnotesize Please go        on to the next page\ldots}}}

\usepackage{cleveref}
\crefname{figure}{figure}{figures}
\crefname{question}{question}{questions}
%==============================================================
\begin{document}

%% \thispagestyle{empty}

\noindent
\begin{minipage}[l]{.5\textwidth}%
\noindent
Name: \underline{\hspace{7cm}}
%\includegraphics[width=1.5\textwidth]{123}
\end{minipage}
\hfill
\begin{minipage}[r]{0.25\textwidth}%
\begin{center}
{%\Large Name of University \\[2pt]
\large BISC 104 -- Quiz 2\\[2pt]} %{(\small Code: Math-506)}  \par}
%  \vspace{0.5cm}
\end{center}
\end{minipage}
\par
\noindent
\uline{Time: 7 minutes   \hfill \normalsize\emph{\underline{Session 03}} \hfill        Maximum Points: 07}
\begin{questions}

\pointsinrightmargin
\pointsdroppedatright
\marksnotpoints
%\marginpointname{mark}
\pointpoints{mark}{marks}
\pointformat{\boldmath\themarginpoints}
\bracketedpoints
\question[01]
\label{Q:perunit}
Mechanical response of a muscle to single electrical stimulus is called \underline{\hspace{3cm}}.
\droppoints

\question[01]
Period of time that elapses between the generation of an active potential and start of muscle contraction is called \underline{\hspace{3cm}} period and it usually remains constant irrespective of the force applied.
\droppoints

\question[01]
Which of the following is \textbf{least likely} to be affected by the potential(voltage) applied to muscle?
\droppoints

\begin{enumerate}[label=\alph*]
\item Active force
\item Passive force
\item Latent period
\end{enumerate}

\question[02]
If stimuli is applied frequently to muscle over a prolonged period, the muscle will
tend to reach \underline{\hspace{3cm}} and the state is called \underline{\hspace{3cm}}:
\droppoints

\begin{enumerate}[label=\alph*]
\item plateau, tetanus
\item latent period, tetanus
\item threshold, latent period
\item fatigue, threshold
\end{enumerate}


\question[02]
\textbf{Decline} in a muscle's ability to maintain a constant force of contraction after prolonged, repetitive stimulation is called :
\droppoints

\begin{enumerate}[label=\alph*]
\item Fatigue
\item Maximal voltage
\item Threshold
\end{enumerate}


 
\end{questions}
\begin{center}
\rule{\textwidth}{1pt}
\end{center}
\end{document} 